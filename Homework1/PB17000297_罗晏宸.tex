\documentclass{article}
\usepackage[UTF8]{ctex}
\usepackage[T1]{fontenc}
\usepackage[utf8]{inputenc}
\usepackage{latexsym}
\usepackage{amsmath}
\usepackage{siunitx}
\usepackage{float}
\title{Homework 1}
\author{PB17000297 罗晏宸}
\date{March 1 2020}

\begin{document}
\maketitle

\section{Exercise 1.8}
架构师面对的一个挑战是, 今天拟定的设计方案可能需要几年的时间进行实施、 验证和测试, 然后才能上市。这就意味着架构师必须提前为几年后的技术进步制定计划。有时, 这很难做到。
\subparagraph{1}
根据摩尔定律观测到的器件发展趋势, 到2025年, 一个芯片上的晶体管数目应当是2005年的多少倍?
\subparagraph{2}
芯片性能的提升也反映了这一趋势。 如果2003年后芯片性能仍以20世纪90年代的相同速度攀升, 到2025年芯片的性能将会是VAX-11/780的多少倍?
\subparagraph{3}
若2003年后以当前的增长速率, 2025年的芯片性能将会是VAX-11/780的多少倍?
\subparagraph{4}
是什么限制了时钟频率的增长速度? 为了提升性能, 架构师现在能用多出来的晶体管做些什么?
\subparagraph{5}
DRAM容量的增长速度也已变缓慢。20年来, DRAM容量每年提高60\%。如果\SI{8}{Gbit} DRAM在2015年首次出现, \SI{16}{Gbit}直到2019年才出现, 那么DRAM容量这段时间的增长速率是多少?

\paragraph{解}
\subparagraph{1}
晶体管密度每年大约增加35\%,因此经过20年:
\begin{equation*}
    (1 + 35\%)^{20} \approx 404.274
\end{equation*}
即2025年时一个芯片上的晶体管数目约是2005年的400倍。
\subparagraph{2}
2003年时处理器性能为Intel Xeon EE \SI{3.2}{GHz},若芯片性能仍以20世纪90年代的速度年增长52\%攀升,则
\begin{equation*}
    \frac{\SI{3.2}{GHz} \times (1 + 52\%)^{22}}{\SI{5}{MHz}} \approx 6.40824 \times 10^{6}
\end{equation*}
即到2025年芯片的性能将会是VAX-11/780的约640万倍。
\subparagraph{3}
若2003年后以当前的增长速率22\%, 则
\begin{equation*}
    \frac{\SI{3.2}{GHz} \times (1 + 22\%)^{22}}{\SI{5}{MHz}} \approx 50827.2
\end{equation*}
即到2025年芯片的性能将会是VAX-11/780的约5万倍。
\subparagraph{4}
2003年之后,功耗和可用指令级并行的限制减慢了单核处理器性能的增长速度,在越来越小的区域内,功率密度不断增加,产生的热量使得散热器工作效率过低,这限制了芯片上晶体管的活动。架构师不再一味增加时钟频率,而是在芯片上放置多个内核。
\subparagraph{5}
设DRAM容量这段时间的年增长速率为$x$,则有
\begin{align*}
    &&\SI{8}{Gbit} \times (1 + x)^{4} &= \SI{16}{Gbit} \\
    \Rightarrow && (1 + x)^{4} &= 2 \\
    \Rightarrow && 1 + x &= \sqrt[4]{2} \\
    \Rightarrow && x &\approx 0.189
\end{align*}
即DRAM容量在2015至2019年间的增长速率为$18.9\%$

\section{Exercise 1.17}
公司刚刚买了一个新的Intel Core i5双核处理器, 你接到针对这一处理器来优化软件的任务。 你将在这个处理器上运行两个应用程序, 但它们的资源需求并不一样。第一个程序需要80\%的资源, 另一个需要20\%的资源。假定对该程序的一部分进行并行化时, 该部分的加速比为2.
\subparagraph{1}
假定第一个应用程序的40\%可以并行化, 那么在隔离运行时, 通过这个应用程序可以实现多大的加速比?
\subparagraph{2}
假定第二个应用程序的99\%可以并行化, 那么在隔离运行时, 通过这个应用程序可以实现多大的加速比?
\subparagraph{3}
假定第一个应用程序的40\%可以并行化, 如果对其实现并行化, 系统总加速比为多少?
\subparagraph{4}
假定第二个应用程序的99\%可以并行化, 如果对其实现并行化, 系统总加速比为多少?

\paragraph{解}
\subparagraph{1}
通过这个应用程序可以实现的加速比为:
\begin{equation*}
    \frac{1}{60\% + \frac{40\%}{2}} = \frac{5}{4} = 1.25
\end{equation*}
\subparagraph{2}
通过这个应用程序可以实现的加速比为:
\begin{equation*}
    \frac{1}{1\% + \frac{99\%}{2}} = \frac{200}{101} \approx 1.98
\end{equation*}
\subparagraph{3}
系统总加速比为:
\begin{equation*}
    \frac{1}{80\% \times (60\% + \frac{40\%}{2}) + 20\%} = \frac{25}{21} \approx 1.19
\end{equation*}
\subparagraph{4}
系统总加速比为:
\begin{equation*}
    \frac{1}{80\% + 20\% \times (1\% + \frac{99\%}{2})} = \frac{1000}{901} \approx 1.11
\end{equation*}

\section{}
假设你的load/store计算机具有以下指令组合:
\begin{table}[H]
    \centering
    \begin{tabular}{|l|c|c|}
    \hline
    Operations & Frequency & No. of Clock Cycle \\ \hline
    ALU ops    & 35\%      & 1                  \\ \hline
    Load       & 25\%      & 2                  \\ \hline
    Store      & 15\%      & 2                  \\ \hline
    Branches   & 25\%      & 3                  \\ \hline
    \end{tabular}
    \end{table}
\subparagraph{1}
计算CPI.
\subparagraph{2}
我们观察到35\%的ALU操作都伴随着一条load指令。 现在我们用一条新的指令来替代这些ALU操作和与之对应的load操作, 这条新指令的执行需要花费1个时钟周期。加入这条指令后, branches需要花费5个时钟周期。请计算新
版本的CPI。
\subparagraph{3}
如果新版本的时钟周期时间是旧版本的1.2倍, 那么哪个版本的CPU执行时
间更短?

\paragraph{解}
\subparagraph{1}
\begin{align*}
    \text{CPI} &= 35\% \times 1 + 25\% \times 2 + 15\% \times 2 + 25\% \times 3 \\
    &= 0.35 + 0.5 + 0.3 + 0.75 \\
    &= 1.9
\end{align*}
\subparagraph{2}
\begin{align*}
    \text{CPI}' =& (35\% - 35\% \times 35\%) \times 1 + (25\% - 35\% \times 35\%) \times 2 + \\
    & \indent (35\% \times 35\%) \times 1 +  15\% \times 2 + 25\% \times 5 \\
    =& 0.2275 + 0.255 + 0.1225 + 0.3 + 1.25 \\
    =& 2.1555
\end{align*}
\subparagraph{3}
两个版本的CPU执行时间之比为:
\begin{equation*}
    \frac{2.1555 \times 1.2}{1.9} \approx 1.361 > 1
\end{equation*}
因此旧版本CPU执行时间更短。
\end{document}