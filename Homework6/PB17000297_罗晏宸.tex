\documentclass{article}
\usepackage[UTF8]{ctex}
\usepackage[T1]{fontenc}
\usepackage[utf8]{inputenc}
\usepackage{latexsym}
\usepackage{amsmath}
\usepackage{amssymb}
\usepackage[table, xcdraw]{xcolor}
\usepackage{listings}
\lstset{
    basicstyle = \ttfamily,
    keywordstyle = \bfseries,
    linewidth = \linewidth,
    % xleftmargin=.1\textwidth,
    % xrightmargin=.1\textwidth,
    breaklines = true,
    breakatwhitespace = true,
    breakautoindent = true,
    frame = none,
    commentstyle = \small\color{gray}
}
\usepackage{siunitx}
\sisetup{
    binary-units = true
}
\usepackage{enumerate}
\usepackage{enumitem}

\title{Homework 6}
\author{PB17000297 罗晏宸}
\date{May 18 2020}

\begin{document}
\maketitle

\section{Exercise 4.9}
假定处理器运行频率为 \SI{700}{\mega\hertz},最大向量长度为 64,载入/存储单元的启动开销为 15 个时钟周期,乘法单元为 8 个时钟周期,加法/减法单元为 5 个时钟周期。在该处理器上进行如下运算,将两个包含单精度复数值的向量相乘:
\begin{lstlisting}[breaklines = false, language=C++]
for (i = 0; i < 300; i++) {
    c_re[i] = a_re[i] * b_re[i] - a_im[i] * b_im[i];
    c_im[i] = a_re[i] * b_im[i] + a_im[i] * b_re[i];
}
\end{lstlisting}
\subparagraph{(1)} 这个内核的运算密度为多少(注:运算密度指运行程序时执行的\textbf{浮点运算数}除以主存储器中\textbf{访问的字节数})?
\subparagraph{(2)} 将此循环转换为使用条带挖掘(Strip Mining)的 VMIPS 汇编代码。
\subparagraph{(3)} 假定采用链接和单一存储器流水线,需要多少次钟鸣?每个复数结果值需要多少个时钟周期(包括启动开销在内)?
\subparagraph{(4)} 如果向量序列被链接在一起,每个复数结果值需要多少个时钟周期(包含开销)?
\subparagraph{(5)} 现在假定处理器有三条存储器流水线和链接。如果该循环的访问过程中没有组冲突,每个结果需要多少个时钟周期?

\paragraph{解}
\subparagraph{(1)} 执行 6 次浮点运算,读 4 个浮点数,写 2 个浮点数,访问 $(4 + 2) \times 4 = 24$ 个字节。内核运算密度为
$$
    \frac{6}{(4 + 2) \times 4} = \frac{1}{4}
$$
\subparagraph{(2)}
\begin{lstlisting}[language={[x86masm]Assembler}, breakindent = 41ex]
        li          $VL, 44         ; 前 44 步操作
        li          $r1, 0          ; 初始化下标
loop:   lv          $v1, a_re + $r1 ; load a_re
        lv          $v3, b_re + $r1 ; load b_re
        mulvv.s     $v5, $v1, $v3   ; a_re * b_re
        lv          $v2, a_im + $r1 ; load a_im
        lv          $v4, b_im + $r1 ; load b_im
        mulvv.s     $v6, $v2, $v4   ; a_im * b_im
        subvv.s     $v5, $v5, $v6   ; a_re * b_re - a_im * b_im
        sv          $v5, c_re + $r1 ; store c_re
        mulvv.s     $v5, $v1, $v4   ; a_re * b_im
        mulvv.s     $v6, $v2, $v3   ; a_im * b_re
        addvv.s     $v5, $v5, $v6   ; a_re * b_im + a_im * b_re
        sv          $v5, c_im + $r1 ; store c_im
        bne         $r1, 0, else    ; 是否首次循环
        addi        $r1, $r1, #176  ; 首次循环
        j loop                      ; 跳转下次循环
else:   addi        $r1, $r1, #256  ; 非首次循环
skip:   blt         $r1, 1200, loop ; 跳转下次循环
\end{lstlisting}

\subparagraph{(3)}
\begin{lstlisting}[language={[x86masm]Assembler}, numbers = left, xleftmargin=.2\textwidth]
mulvv.s     lv
lv          mulvv.s
subvv.s     sv
mulvv.s     lv      ; load 下一个向量
mulvv.s     lv      ; load 下一个向量
addvv.s     sv
\end{lstlisting}
共需要 6 次钟鸣,每个复数结果值需要的时钟周期为
$$
    \frac{6 \times 64 + 15 \times 6 + 8 \times 4 + 5 \times 2}{2 \times 64} = \frac{516}{128} = \frac{129}{32} = 4.03125
$$
\subparagraph{(5)}
\begin{lstlisting}[language={[x86masm]Assembler}, numbers = left, xleftmargin=.2\textwidth]
mulvv.s
mulvv.s
subvv.s     sv
mulvv.s
mulvv.s     lv
addvv.s     sv  lv  lv  lv ; load 下一个向量
\end{lstlisting}
共需要 6 次钟鸣,因此每个复数结果值需要的时钟周期为$4.03125$与上相同。

\section{Exercise 4.16}
假定一个虚设 GPU 具有以下特性:
\begin{enumerate}[leftmargin = .1\linewidth, rightmargin = .1\linewidth, label=$\bullet$]
    \item 时钟频率为 \SI{1.5}{\giga\hertz}
    \item 包含 16 个 SIMD 处理器,每个处理器包含 16 个单精度浮点单元
    \item 片外存储器带宽为 \SI{100}{{\giga\byte}\per\second}
\end{enumerate}
\subparagraph{(1)} 不考虑存储器带宽,假定所有存储器延迟可以隐藏,则这一 GPU 的峰值单精度浮点吞吐量为多少 \si{GFLOP\per\second}?
\subparagraph{(2)} 在给定存储器带宽限制下,这一吞吐量是否可持续?
\paragraph{解}
\subparagraph{(1)}
$$
    \SI{1.5}{\giga\hertz} \times 16 \times 16 = \SI{384}{GFLOP\per\second}
$$
\subparagraph{(2)}
每个单精度运算需要读 2 个操作数,写 1 个操作数,访问$(2 + 1) \times 4 = 12$个字节,需要$\SI{12}{\byte} \times \SI{384}{GFLOP\per\second} = \SI{4608}{{\giga\byte}\per\second}$,因此吞吐量不可持续。

\section{Exercise 4.13}
假定有一种包含 10 个 SIMD 处理器的 GPU 体系结构。每条 SIMD 指令的宽度为 32,每个 SIMD 处理器包含 8 个车道,用于执行单精度运算和载入/存储指令,也就是说,每个非分岔 SIMD 指令每 4 个时钟周期可以生成 32 个结果。假定内核的分岔分支将导致平均 80\%的线程为活动的。假定在所执行的全部 SIMD 指令中,70\%为单精度运算,20\%为载入/存储。由于并不包含所有存储器延迟,所以假定SIMD 指令平均发射率为 0.85。假定 GPU 的时钟速度为 \SI{1.5}{\giga\hertz}。
\subparagraph{(1)} 计算这个内核在这个 GPU 上的吞吐量,单位为 \si{GFLOP\per\second}。
\subparagraph{(2)} 对于以下改进中的\textbf{每一项},吞吐量的加速比为多少?
\begin{enumerate}[leftmargin = .2\linewidth, rightmargin = .2\linewidth, label=\textcircled{\arabic*}]
    \item 将单精度车道数增大至 16
    \item 将 SIMD 处理器数增大至 15(假定这一改变不会影响所有其他性能度量,代码会扩展到增加的处理器上)。
    \item 添加缓存可以有效地将存储器延迟缩减 40\%,这样会将指令发射率增加至0.95。
\end{enumerate}

\paragraph{解}
\subparagraph{(1)}
$$
    \SI{1.5}{\giga\hertz} \times 80\% \times 85\% \times 70\% \times 10 \times 8 = \SI{57.12}{GFLOP\per\second}
$$
\subparagraph{(2)}
\begin{enumerate}[leftmargin = .2\linewidth, rightmargin = .2\linewidth, label=\textcircled{\arabic*}]
    \item 加速比为
          \begin{align*}
                & \frac{ \SI{1.5}{\giga\hertz} \times 80\% \times 85\% \times 70\% \times 10 \times 16}{\SI{57.12}{GFLOP\per\second}} \\
              = & \frac{\SI{114.24}{GFLOP\per\second}}{\SI{57.12}{GFLOP\per\second}}                                                  \\
              = & 2
          \end{align*}
    \item 加速比为
          \begin{align*}
                & \frac{ \SI{1.5}{\giga\hertz} \times 80\% \times 85\% \times 70\% \times 15 \times 8}{\SI{57.12}{GFLOP\per\second}} \\
              = & \frac{\SI{85.68}{GFLOP\per\second}}{\SI{57.12}{GFLOP\per\second}}                                                  \\
              = & 1.5
          \end{align*}
    \item 加速比为
          \begin{align*}
                & \frac{ \SI{1.5}{\giga\hertz} \times 80\% \times 95\% \times 70\% \times 10 \times 8}{\SI{57.12}{GFLOP\per\second}} \\
              = & \frac{\SI{63.84}{GFLOP\per\second}}{\SI{57.12}{GFLOP\per\second}}                                                  \\
              = & \frac{19}{17}                                                                                                      \\
              = & 1.118
          \end{align*}
\end{enumerate}
\end{document}